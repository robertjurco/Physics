\documentclass{article}
\usepackage[a4paper, landscape, top = 2cm, bottom = 2cm, left = 2cm, right = 2cm]{geometry}
\usepackage{multicol}
\usepackage{bm}
\usepackage{xcolor}

\usepackage{graphicx}\usepackage{lmodern}

\usepackage[sc]{mathpazo} % Use the Palatino font
\usepackage[utf8]{inputenc}
\usepackage[T1]{fontenc}
\linespread{1.05} % Line spacing - Palatino needs more space between lines
\usepackage{microtype} % Slightly tweak font spacing for aesthetics
\setlength\parindent{0pt}
%\usepackage{textgreek}

\usepackage{fancyhdr} % Headers and footers
\pagestyle{fancy} % All pages have headers and footers
\fancyhead{} % Blank out the default header
\fancyfoot{} % Blank out the default footer
\fancyhead[C]{Róbert Jurčo \hfill Electromagnetism and electrodynamics cheatsheet} % Custom header text
\fancyfoot[C]{\thepage} % Custom footer text

\usepackage{amsmath, amssymb, amsfonts, amsthm, wasysym}
\newcommand{\N}{\mathbb{N}}
\newcommand{\Z}{\mathbb{Z}}
\newcommand{\Q}{\mathbb{Q}}
\newcommand{\R}{\mathbb{R}}
\newcommand{\C}{\mathbb{C}}

\renewcommand\d{\mathrm d}

\DeclareMathOperator{\arccot}{arccot}
\DeclareMathOperator{\arcsinh}{arcsinh}
\DeclareMathOperator{\arccosh}{arccosh}
\DeclareMathOperator{\arctanh}{arctanh}
\DeclareMathOperator{\arccoth}{arccoth}

% TENSOR
\usepackage{amsmath}

\DeclareFontFamily{OMS}{oasy}{\skewchar\font48 }
\DeclareFontShape{OMS}{oasy}{m}{n}{%
         <-5.5> oasy5     <5.5-6.5> oasy6
      <6.5-7.5> oasy7     <7.5-8.5> oasy8
      <8.5-9.5> oasy9     <9.5->  oasy10
      }{}
\DeclareFontShape{OMS}{oasy}{b}{n}{%
       <-6> oabsy5
      <6-8> oabsy7
      <8->  oabsy10
      }{}
\DeclareSymbolFont{oasy}{OMS}{oasy}{m}{n}
\SetSymbolFont{oasy}{bold}{OMS}{oasy}{b}{n}

\DeclareMathSymbol{\smallleftarrow}     {\mathrel}{oasy}{"20}
\DeclareMathSymbol{\smallrightarrow}    {\mathrel}{oasy}{"21}
\DeclareMathSymbol{\smallleftrightarrow}{\mathrel}{oasy}{"24}

\newcommand{\tensor}[1]{\overset{\scriptscriptstyle\smallleftrightarrow}{#1}}
% END TENSOR

\begin{document}
\begin{multicols}{3}

    \centering{\large\textbf{I. VECTOR ANALYSIS}}

    \begin{enumerate}
        \item $\bm a\cdot(\bm b\times\bm c)=\bm b\cdot(\bm c\times\bm a)=\bm c\cdot(\bm a\times\bm b)$
        \item $\bm a\times(\bm b\times\bm c)=\bm b(\bm a\cdot\bm c)-\bm c(\bm a\cdot\bm b)$
        \item $\nabla(fg)=f\nabla g+g\nabla f$
        \item $\nabla\cdot(f\bm F)=f\nabla\cdot\bm F+\bm F\cdot\nabla f$
        \item $\nabla\times(f\bm F)=f\nabla\times\bm F-\bm F\times\nabla f$
        \item $\nabla(\bm F_1\cdot\bm F_2)=(\bm F_1\cdot\nabla)\bm F_2+(\bm F_2\cdot\nabla)\bm F_1+\bm F_1\times(\nabla\times\bm F_2)+\bm F_2\times(\nabla\times\bm F_1)$
        \item $\nabla\cdot(\bm F_1\times\bm F_2)=\bm F_2\cdot(\nabla\times\bm F_1)-\bm F_1\cdot(\nabla\times\bm F_2)$
        \item $\nabla\times(\bm F_1\times\bm F_2)=(\bm F_2\cdot\nabla)\bm F_1-(\bm F_1\cdot\nabla)\bm F_2+\bm F_1\nabla\cdot\bm F_2-\bm F_2\nabla\cdot\bm F_1$
        \item $\nabla\times\bm R=0$, $\nabla\cdot\bm R=\partial_x R_x+\partial_y R_y+\partial_z R_z$\\ $\nabla R=\bm R/R$, $\nabla R^n=nR^{n-2}\bm R$, $\nabla(\bm c\cdot\bm R)=\bm c$\\
        $\nabla\times(\bm R/R)=0$
        \item $\nabla((\bm c\cdot\bm R)/R)=(R^2\bm c-(\bm c\cdot\bm R)\bm R)/R^3$.
        \item Gauss's theorem: $\int_V(\nabla\cdot\bm a)\;\d V=\oint_{\partial V}\bm a\cdot\d\bm S$
        \item Stokes's theorem: $\int_S(\nabla\times\bm a)\cdot\d\bm S=\oint_{\partial S}\bm a\cdot\d\bm l$
        \item Green's theorem:\\$\oint_{\partial V}(\varphi\nabla\psi-\psi\nabla\varphi)\cdot\d\bm S=\int_V(\varphi\Delta\psi-\psi\Delta\varphi)\;\d V$
    \end{enumerate}

    \centering{\large\textbf{II. ELECTROSTATICS}}

    \begin{enumerate}
        \item Coulomb's law: $\bm F=\frac{1}{4\pi\epsilon}\frac{Q_1Q_2}{r^3}\bm r$, $\Pi=\frac{1}{4\pi\epsilon}\frac{Q_1Q_2}{r}$, Kepler's law are applicable
        \item Intensity and potential of el. field\\ $\bm E=\frac{1}{4\pi\epsilon}\frac{Q}{r^3}\bm r$, $\varphi=\frac{1}{4\pi\epsilon}\frac{Q}{r}$, $\bm E=-\nabla\varphi$
        \item Gauss's law: $\oint_{\partial V}\bm E\cdot\d\bm S=Q/\epsilon_0$, $\nabla\cdot\bm E=\rho/\epsilon_0$
        \item Conservative force: $\oint_{\partial S}\bm E\cdot\d\bm l=0$
        \item There is no stable system of charges in the\\vacuum.
        \item Poisson equation: $\nabla\varphi=-\rho/\epsilon_0$,\\Laplace equation free space $\nabla\varphi=0$.
        \item $\varphi(\bm r)=\frac{1}{4\pi\epsilon_0}\int_{V'}\frac{\rho(\bm {r'})}{|\bm r-\bm{r'}|}\d V'$\\$\bm E(\bm r)=\frac{1}{4\pi\epsilon_0}\int_{V'}\frac{(\bm r-\bm{r'})\rho(\bm {r'})}{|\bm r-\bm{r'}|^3}\d V'$
        \item Multipole expansion: $$\varphi(\bm r)=\frac{1}{4\pi\epsilon_0}\frac{1}{r}\sum_{i=1}^\infty\int_{V'}\left[\frac{r'}{r}\right]^i\rho(\bm {r'})P_i(\cos\alpha)\d V'$$
        where $P_i(x)$ is $i$-th Legendre polynomial and $\cos\alpha$ is the angle between $\bm r'$ and $\bm r$.
        \item Electric dipole moment $\bm p=Q\bm l=\sum Q_i\bm r_i$
        \item El. dipole: $\varphi(\bm r)=\frac{1}{4\pi\epsilon}\frac{\bm p\cdot\bm r}{r^3}$\\$\bm E(\bm r)=\frac{1}{4\pi\epsilon}\left[\frac{3\bm r(\bm p\cdot\bm r)}{r^5}-\frac{\bm p}{r^3}\right]$
        \item z-axis of dipole, $\tan\theta=x/z$:\\ $E_x=\frac{P}{4\pi\epsilon_0}\frac{3\sin\theta\cos\theta}{r^3}$, $E_z=\frac{P}{4\pi\epsilon_0}\frac{3\cos^2\theta-1}{r^3}$
        \item Energy of dipole: $W=-\bm p\cdot\bm E$
        \item Force acting on a dipole: $F_i=p_j\partial_i E_j$,\\$\bm F=(\bm p\cdot\nabla)\bm E$, for constant dipole $\bm F=\nabla(\bm p\cdot\bm E)$
        \item Torque acting on a dipole: $\bm\tau=\bm p\times\bm E$
    \end{enumerate}


    \centering{\large\textbf{III. ELECTRIC FIELDS IN MATTER}}

    \begin{enumerate}
        \item $\rho_{total}=\rho_{free}+\rho_{bounded}$
        \item Polarization $\bm P=N\bm p=N\tensor\beta\bm E=\epsilon_0\tensor\chi\bm E$
        \item Tensor of suspecibility $\tensor\chi=N\tensor\beta/\epsilon_0$\\
        where $\tensor\beta$ is atomic polarizability\\
        \begin{tabular}{|c|c|c|}
            \hline
             & izotrop. & unizotrop. \rule{0pt}{2ex}\\\hline
            homogenous & $\chi$ & $\tensor\chi$ \rule{0pt}{2ex}\\\hline
            non-homog. & $\chi(\bm r)$ & $\tensor\chi(\bm r)$ \rule{0pt}{2ex}\\\hline
        \end{tabular}
        \item Electric induction: $\bm D=\epsilon_0\bm E+\bm P=\tensor\epsilon\bm E$, where $\tensor\epsilon=\epsilon_0(1+\tensor\chi)$
        \item $\nabla\cdot\bm P=-\rho_{b}/\epsilon_0$, $\oint_{\partial V}\bm P\cdot\d\bm S=-Q_{b}/\epsilon_0$
        \item BOunded charge: $Q_{b}=\bm P\cdot\bm S$, $\sigma_b=\bm p\cdot\bm n$
        \item Gauss's law: $\oint_{\partial V}\bm D\cdot\d\bm S=Q_{free}$, $\nabla\cdot\bm D=\rho_{free}$
        \item Conservative force: $\oint_{\partial S}\bm E\cdot\d\bm l=0$
        \item Boundary of dielectrics:\\$E_{2t}=E_{1t}$, $D_{2n}-D_{1n}=\sigma_{free}$, where $\sigma_{free}$ is free charge surface density on the boundary.
        \item Inside a conductor $\bm E=0$.
        \item Outside of conductor the electric field lines are perpendicular to the surface, beggining or ending at charges on the surface.
        \item Induced charges resides entirely on the surface of the conductor and the total charge is usually zero. It is nonzero if conductor is grounded or infinite.
        \item Whatever be the charge and field configuration outside, any cavity in a conductor remains shielded from outside electric influence.
        \item Method of image charges.

        \item Clausiuss-Mossotti eq. $\frac{\epsilon-1}{\epsilon+2}=\frac{N\alpha}{3\epsilon_0}=\frac{\chi}{\chi+3}$, \\where $\alpha$ is molecular polarizability.
    \end{enumerate}


    \centering{\large\textbf{IV. MAGNETOSTATICS}}

    \begin{enumerate}

    \end{enumerate}

    \centering{\large\textbf{V. MAGNETIC FIELDS IN MATTER}}

    \begin{enumerate}

    \end{enumerate}

    \centering{\large\textbf{VI. MAGNETOSTATICS}}

    \begin{enumerate}

    \end{enumerate}

    \centering{\large\textbf{VII. ELECTRIC CURRENT}}

    \begin{enumerate}

    \end{enumerate}

    \centering{\large\textbf{II. CONVERSATION LAWS}}

    \begin{enumerate}

    \end{enumerate}

    \centering{\large\textbf{VIII. NONSTATIONARY FIELDS}}

    \begin{enumerate}

    \end{enumerate}

    \centering{\large\textbf{IX. TRANSPORTNI JEVY}}

    \begin{enumerate}

    \end{enumerate}

    \centering{\large\textbf{X. ELECTROMAGNETIC WAVES}}

    \begin{enumerate}

    \end{enumerate}

    \centering{\large\textbf{XI. RADIATION}}

    \begin{enumerate}

    \end{enumerate}

\end{multicols}
\end{document}
