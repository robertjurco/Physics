\documentclass{article}
\usepackage[a4paper, landscape, top = 2cm, bottom = 2cm, left = 2cm, right = 2cm]{geometry}
\usepackage{multicol}
\usepackage{bm}
\usepackage{xcolor}

\usepackage{graphicx}\usepackage{lmodern}

\usepackage[sc]{mathpazo} % Use the Palatino font
\usepackage[utf8]{inputenc}
\usepackage[T1]{fontenc}
\linespread{1.05} % Line spacing - Palatino needs more space between lines
\usepackage{microtype} % Slightly tweak font spacing for aesthetics
\setlength\parindent{0pt}
%\usepackage{textgreek}

\usepackage{fancyhdr} % Headers and footers
\pagestyle{fancy} % All pages have headers and footers
\fancyhead{} % Blank out the default header
\fancyfoot{} % Blank out the default footer
\fancyhead[C]{Róbert Jurčo \hfill Special relativity cheatsheet} % Custom header text
\fancyfoot[C]{\thepage} % Custom footer text

\usepackage{amsmath, amssymb, amsfonts, amsthm, wasysym}
\newcommand{\N}{\mathbb{N}}
\newcommand{\Z}{\mathbb{Z}}
\newcommand{\Q}{\mathbb{Q}}
\newcommand{\R}{\mathbb{R}}
\newcommand{\C}{\mathbb{C}}

\renewcommand\d{\mathrm d}

\DeclareMathOperator{\arccot}{arccot}
\DeclareMathOperator{\arcsinh}{arcsinh}
\DeclareMathOperator{\arccosh}{arccosh}
\DeclareMathOperator{\arctanh}{arctanh}
\DeclareMathOperator{\arccoth}{arccoth}


\begin{document}
\begin{multicols}{3}

    \centering{\large\textbf{I. PRINCIPES OF SPECIAL RELATIVITY}}

    \begin{enumerate}
        \item The laws of physics are invariant in all inertial frames of reference.
        \item The speed of light in vacuum is the same for all observers, regardless of the motion of the light source or observer.
    \end{enumerate}


    \centering{\large\textbf{II. LORENTZ TRANSFORMATION}}

    \begin{enumerate}
        \item {\color{red} Michelson-Morley experiment.}
        \item Lorentz factor $\gamma=(1-\frac{v^2}{c^2})^{-1/2}$ and relatative velocity $\beta=v/c$.
        \item Lorentz transformation between frames $S'$ ad $S$ with relative vel. $v$ is $x'^\mu=\Lambda^\mu_{~\nu} x^\nu$. Where Lorentz boost $\Lambda$ takes form\\
        $\begin{pmatrix}
            ct'\\ x'\\ y' \\ z'
        \end{pmatrix}
        =\begin{pmatrix}
            \gamma & -\beta\gamma & 0 & 0\\
            -\beta\gamma & \gamma & 0 & 0\\
            0 & 0 & 1 & 0\\
            0 & 0 & 0 & 1
        \end{pmatrix}
        \begin{pmatrix}
            ct\\ x\\ y \\ z
        \end{pmatrix}$
        \item $\Lambda^\alpha_{~\nu}\Lambda^{~\mu}_\alpha=\delta^\mu_\nu$ and $(\Lambda^\mu_{~\nu})^{-1}=\Lambda^{~\mu}_\nu$
        \item Lorentz boost in direction $\vec n$ of magnitude $v$:\\
            $ct'=\gamma\left(ct-\beta\vec n\cdot\vec r\right)$,\\
            $\vec r'=\left[\vec r-(\vec r\cdot\vec n)\vec n\right]+\gamma\left[(\vec r\cdot\vec n)\vec n-\beta ct\vec n\right]$\\
            where we used decomposition of $\vec r$,\\ $\vec r^{~\parallel}=(\vec r\cdot\vec n)\vec n$ and $\vec r^{~\perp}=\vec r-(\vec r\cdot\vec n)\vec n$.
        \item $\beta=\tanh\xi$, $\gamma=\cosh\xi$ and $\beta\gamma=\sinh\xi$ where $\xi$ is rapidity.
        \item
        $\begin{pmatrix}
            ct'\\ x'\\ y' \\ z'
        \end{pmatrix}
        =\begin{pmatrix}
            \cosh\xi & -\sinh\xi & 0 & 0\\
            -\sinh\xi & \cosh\xi & 0 & 0\\
            0 & 0 & 1 & 0\\
            0 & 0 & 0 & 1
        \end{pmatrix}
        \begin{pmatrix}
            ct\\ x\\ y \\ z
        \end{pmatrix}$
        \item Time dilation $t=t_0\gamma$.
        \item Length contraction $l=l_0/\gamma$.
        \item Relativity of conplacemency $\Delta x=v\Delta t$.
        \item Relativity of simultaneity $c\Delta t'=-\beta\gamma\Delta x$.
        \item Transformation of 3-dim velocity: if $v$ is along $x$ axis:
            $w'_x=\frac{w_x-v}{1-\frac{v}{c^2}w_x}$,
            $w'_y=\frac{1}{\gamma}\frac{w_y}{1-\frac{v}{c^2}w_x}$.
        \item Addition of vel. $u$ and $v$: $w=(u+v)\left(1+\frac{uv}{c^2}\right)$,
        \item Hybrid velocity (distance in not-moving frame $S$ but proper time) can be greater then $c$. If $\vec w$ is velocity in $S$, then hybrid vel. is $\frac{\d x}{\d t'}=\vec w\gamma$.
        \item Space time interval $\d s^2=-c^2\d t^2+\d x^2+\d y^2+\d z^2$.
        \item Space-time interval is invariant under Lorentz transformation.
    \end{enumerate}


    \centering{\large\textbf{III. SPACETIME}}

    \begin{enumerate}
        \item Vectors transform as $\d x'^\mu=\frac{\partial x'^\mu}{\partial x^\nu}\d x^\nu$ (choosen so bc. we want $\d s$ as a vector). Especially $\d x'^\mu=\Lambda^\mu_{~\nu}\d x^\nu$.
        \item Covectors transform as $\frac{\partial}{\partial x'^\mu}=\frac{\partial x^\nu}{\partial x'^\mu}\frac{\partial}{\partial x^\nu}$. Especially $\partial'_\mu=\Lambda_\mu^{~\nu}\partial_\nu$.
        \item $\partial_\nu\d x^\mu=\d x^\mu\partial_\nu=\delta^\mu_\nu=\d x'^\mu\partial'_\nu=\partial'_\nu\d x'^\mu$
        \item Metric tensor $g_{\mu\nu}$ (a value of metric tensor field in the given point) is a billinear, symmetric, nondegenerate tensor.
        \item Metric tensor defines invariant dot product\\ $g'_{\mu\nu}V'^\mu V'^\nu=g_{\mu\nu}V^\mu V^\nu$.
        \item $\d s^2=g_{\mu\nu}\d x^\mu\d x^\nu$
        \item Metric tensor is invariant under Lorentz transformation $g_{\mu\nu}=g_{\alpha\beta}\Lambda^\alpha_{~\mu}\Lambda^\beta_{~\nu}$, but gen. transforms as  $g'_{\mu\nu}=\frac{\partial x^\alpha}{\partial x'^\mu}\frac{\partial x^\beta}{\partial x'^\nu}g_{\alpha\beta}$ (relation of ortogonality).
        \item Sylvester's law of inertia: Signature\\ (number of +, -, 0 eigenvalues) of matrix is invariant under change of basis.
        \item Inverse metric tensor: $g^{\mu\nu}:=\left(g^{-1}\right)^{\mu\nu}$,\\ $g^{\mu\alpha}g_{\alpha\nu}=g_{\nu\alpha}g^{\alpha\mu}=\delta^\mu_\nu$
        \item Lowering/rising inds. $g_{\mu\nu}A^\nu=A_\mu$, $g^{\mu\nu}A_\nu=A^\mu$
        \item $g_{\mu\nu}V^\mu W^\nu=V^\mu g_{\mu\nu}W^\nu=W^\nu V^\mu g_{\mu\nu}=V_\nu W^\nu=W^\nu V_\nu$
        \item Partial gradient transforms as a covector only under linear transformations:\\
            $\frac{\partial V'^\mu}{\partial x'^\nu}=\frac{\partial^2 x'^\mu}{\partial x^\alpha\partial x^\beta}\frac{\partial x^\beta}{\partial x'^\nu}V^\alpha+\frac{\partial x'^\mu}{\partial x^\alpha}\frac{\partial V^\alpha}{\partial x^\beta}\frac{\partial x^\beta}{\partial x'^\nu}$.
        \item Dot product $g_{\mu\nu}V^\mu V^\nu$ is invariant $\Leftrightarrow$ trnasformation is linear and satisfy relation of orthogonality.
        \item Metrix tensor of Minkowski spacetime \\cartesian: $\eta_{\mu\nu}=\text{diag}(-1,1,1,1)$\\cylindrical: $\eta_{\mu\nu}=\text{diag}(-1,1,\rho^2,1)$\\spherical: $\eta_{\mu\nu}=\text{diag}(-1,1,r^2,r^2\sin^2\theta)$
        \item Dot product of two timelike vectors oriented into the future is always negatve.
        \item Levi-Civita tensor $\epsilon_{ijk\ldots}$ is $1$ for even permutation of $\{1,2,3,\ldots\}$, $-1$ for odd permutation and zero otherwise. $\epsilon^{ijk\ldots}$ has opposite signs.
        \item $\epsilon^{\alpha\beta\gamma\delta}=\eta^{\alpha i}\eta^{\beta j}\eta^{\gamma k}\eta^{\delta l}\epsilon_{ijkl}$
    \end{enumerate}


    \centering{\large\textbf{III. SPACETIME DIAGRAMS}}

    \begin{enumerate}
        \item
    \end{enumerate}

    \centering{\large\textbf{IV. RELATIVISTIC MECHANICS}}

    \begin{enumerate}
        \item Proper time $\tau$ along timelike wolrd line - time in a frame of the system.
        \item Experimentaly is measured only the proper time.
        \item $\d s=c\d\tau$, $\d t=\gamma\d\tau$ hence $\Delta\tau=\int\frac{\d s}{c}=\int\frac{\d t}{\gamma(t)}$
        \item 4-velocity: $u^\mu=\frac{\d x^\mu}{\d\tau}$ and $u'^\mu=\Lambda^\mu_{~\nu}u^\nu$
        \item Minkowsi spacetime $\eta_{\mu\nu}u^\mu u^\nu=-c^2$.
        \item $u^\mu=\gamm\left(c,\vec v\right)^T$, where $\vec v=\d\vec x/\d t$ is classical 3-vel.
        \item 4-acceleration: $a^\mu=\frac{\d u^\mu}{\d\tau}$ and $a'^\mu=\Lambda^\mu_{~\nu}a^\nu$
        \item $a^\mu\perp u^\mu$ for every point of timelike world line, $\eta_{\mu\nu}a^\mu u^\nu=0$
        \item If $a^\mu=0$ in one ref. frame, then it is in all ref. frames if transformation is linear.
        \item Invariant mass $m_0$ is equal to the mass in the rest frame of the point mass.
        \item Relativistic mass $m=m_0\gamma$.
        \item Conservation of \textbf{3-dim} lin. momentum works the same way as in classical mechanics, $\sum_{before} m_iv_i=\sum_{after} m_iv_i$, but with relativistic masses and with relat. addition of velocities.
        \item Inavriant mass of resultant body after inealsitc collision is bigger then invariant masses of coliding bodies.
        \item 4-momentum: $p^\mu=m_0u^\mu=\left(c,\vec p\right)^T$
        \item 4-force: $F^\mu=\frac{\d p^\mu}{\d\tau}=\gamma\frac{\d p^\mu}{\d t}=\gamma\left(c\frac{\d m}{\d t},\vec f\right)^T$, where $\vec f=\frac{\d\vec p}{\d t}$ is 3-force.
        \item $\frac{\d p^\mu}{\d\tau}=\frac{\d m_0}{\d\tau}u^\mu+m_0a^\mu$
        \item 3-force: $\frac{\d\vec p}{\d t}=m\vec a+\left(m_0\gamma^2\frac{\vec v\cdot\vec a}{c^2}+\frac{\d m_0}{\d t}\right)\gamma\vec v$
        \item $\frac{\d\gamma}{\d t}=\gamma^3\frac{\vec v\cdot\vec a}{c^2}$
        \item Mag. field (Lorenz force) $\vec v\cdot\vec a=0$ and $\frac{\d m_0}{\d t}=0$.
        \item 4-forces for which $\frac{\d m_0}{\d t}=0$ are spacelike.
        \item If $\d m_0/\d t=0$ then $\vec f_\perp=m\vec a_\perp$ and $\vec f_\parallel=m\gamma^2\vec a_\parallel$.
        \item $\eta_{\mu\nu}F^\mu u^\nu=-c^2\d m_0/\d\tau$.
        \item $c^2\d m=\frac{1}{\gamma}c^2\d m_0+\vec f\cdot\d\vec r$.
        \item $E=mc^2=m_0c^2+T=\sqrt{(m_0c^2)^2+(pc)^2}$, where $p=|\vec p|$.
        \item Mass shell (3-dim hyperboloid):\\ $m^2c^2-p^2=m_0^2c^2$
        \item If $p^2\ll m_0^2c^2$ then $E=m_0c^2+p^2/2m_0$.
        \item For masseles objects $|\vec v|=c$ and $E=T=pc$, or in wave-language $E=\hbar\omega$.
        \item {\color{red} Covariant rel. for $E$}
        \item $v>c$, $v=c$, $v<c$ is invariant.
    \end{enumerate}


    \centering{\large\textbf{V. ELECTRODYNAMICS IN VACUUM}}

    \begin{enumerate}
        \item Charge density $\rho=\rho_0\gamma$
        \item 4-current: $J^\mu=\left(\rho c,\vec J\right)^T=\rho_0u^\mu$, where $\vec J=\rho\vec v$
        \item $\eta_{\mu\nu}J^\mu J^\nu=-c^2\rho_0^2$.
        \item Gradient: $\partial_\mu=\left(\frac{1}{c}\frac{\partial}{\partial t},\nabla\right)^T$, $\partial^\mu=\frac{\partial}{\partial x_\mu}$, $\partial_\mu=\frac{\partial}{\partial x^\mu}$, $\partial_\mu\partial^\nu=\Delta-\frac{1}{c^2}\frac{\partial^2}{\partial t^2}$.
        \item 4-potential: $A^\mu=\left(\frac{\varphi}{c},\vec A\right)^T$.
        \item EM-field tensor: $F^{\mu\nu}=\partial^\mu A^\nu-\partial^\nu A^\mu$ or\\$F_{\mu\nu}=\partial_\mu A_\nu-\partial_\nu A_\mu$.
        \item $F_{\mu\nu}=
            \begin{pmatrix}
                0 & -E_x/c & -E_y/c  & -E_z/c\\
                E_x/c & 0 & B_z & -B_y\\
                E_y/c & -B_z & 0 & B_x\\
                E_z/c & B_y & -B_x & 0
            \end{pmatrix}$\\
            Rising indicies just swaps the signs of zeroth row and column.
        \item Continuity eq.: $\partial_\mu J^\mu=0$.
        \item 1st. series of Maxwell eq. $\partial_\nu F^{\mu\nu}=\mu J^\mu$.
        \item 2st. series of Maxwell eq. $\partial_{\left[\alpha\right }F_{\left \beta\gamma\right]}=0$.
        \item Maxwell eqs. are invariant under any gauge transformation $\tilde A_\mu=A_\mu+\partial_\mu\chi$.
        \item Lorenz gauge: $\partial_\mu\tilde A^\mu=0$, hence\\ $\partial_\mu\partial^\mu\chi=-\partial_\mu A^\mu$.
        \item In Lorenz gauge 1st. s. of Maxwell eq. gives us wave equation $\partial_\mu\partial^\mu A^\nu=-\mu J^\nu$.
        \item $\partial_\mu\partial^\mu F_{\alpha\beta}=\mu\left(\partial_\beta J_\alpha-\partial_\alpha J_\beta\right)$
        \item Hodge dual elmag. tensor: $(*F)^{\mu\nu}=\frac{1}{2}\epsilon^{\mu\nu\rho\sigma}F_{\rho\sigma}$.
        \item $(*F)^{\mu\nu}=
            \begin{pmatrix}
                0 & -B_x & -B_y & -B_z\\
                B_x & 0 & E_z/c & -E_y/c\\
                B_y & -E_z/c & 0 & E_x/c\\
                B_z & E_y/c & -E_x/c & 0
            \end{pmatrix}$\\
        \item {\color{red} signs notation chcek} $F_{\rho\sigma}F^{\rho\sigma}=2B^2-2E^2/c^2$, $(*F)_{\rho\sigma}(*F)^{\rho\sigma}=-2B^2+2E^2/c^2$, $F_{\rho\sigma}(*F)^{\rho\sigma}=\frac{4}{c}\vec E\cdot\vec B$.
        \item {\color{red} signs notation chcek} Using notation where Levi-Civita symbol has indicies up $\epsilon^{ijkl}$, $*F$ will have all signs opposite and $F_{\rho\sigma}(*F)^{\rho\sigma}=-\frac{4}{c}\vec E\cdot\vec B$.
    \end{enumerate}


    \centering{\large\textbf{\color{red} OO. SOME OTHER SHIT}}

    \begin{enumerate}
        \item Thomas precession
        \item Relativistic disk
        \item Bell's spaceship paradox
        \item Ehrenfest paradox

    \end{enumerate}


\end{multicols}
\end{document}
